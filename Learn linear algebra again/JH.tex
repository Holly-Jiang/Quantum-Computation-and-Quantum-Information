\documentclass[UTF8]{ctexart}
\begin{document}
	\title{\textbf{Propositional SAT Solving}\\[1ex]\begin{large}
		\end{large}}
	\author{软件工程学院 51194501050 蒋慧}
	\maketitle


\section{行列式}
	\subsection{基础定义}
	\par{全排列,逆序,对换(对换会改变排列的奇偶性)}
	\par{二阶行列式:对角线相乘}
	\par{三阶行列式:对角线法则,平行于主对角线为正,平行于副对角线为负,每项中三个元素都位于不同的行不同的列,三个元素的行标都是123 列标都是123的全排列,刚好六种,各项的正负与列标的奇偶排列有关,对角线法则仅适用于二三阶。}
	\par{n阶行列式定义如三阶行列式定义,三个直观分解}
	\par{}
	\par{}
	\par{}
	\par{}
	\par{}
	\par{}
	\par{}
	\par{}
	\par{}
	\subsection{}

	\par{}

\section{向量组的线性相关性}
	\subsection{}
	
	\subsection{应试教育与素质教育对比分析}

	\par{}
\section{线性方程组}
	\subsection{}
	
	\subsection{应试教育与素质教育对比分析}

	\par{}
\section{特征值与特征向量}
	\subsection{}
	
	\subsection{应试教育与素质教育对比分析}

	\par{}

\section{二次型}
	\subsection{}
	
	\subsection{应试教育与素质教育对比分析}

	\par{}

	
\end{document}