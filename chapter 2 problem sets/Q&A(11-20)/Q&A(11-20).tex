\documentclass[UTF8]{ctexart}
\usepackage{physics}
\usepackage{amsmath}
\usepackage{geometry}
 \usepackage{indentfirst} 
 %\mathinner{\langle a | }\quad \mathinner{ | b \rangle}
% \def\bra#1{\mathinner{\langle{#1}}


 \setlength{\parindent}{2em}
\geometry{a4paper,scale=0.8}
\begin{document}
	\title{\textbf{Q\&A(2.11-2.20)}\\[1ex]\begin{large}
		\end{large}}
	\author{Luotingyu\quad Jianghui}
	\date{11th September 2019 }
	\maketitle
\begin{quote}
\textbf{Exercise 2.11: (Eigenedecomposition of the Pauli matrices)} Find the eigenvectors, eigenvalues, and diagonal representations of the Pauli matrices $X, Y, Z$. \\ \\
\textbf{Answer:}\\ \\
	$X=\begin{bmatrix}0& 1\\1& 0\end{bmatrix}$,	
	$Y=\begin{bmatrix}0& -\it{i}\\i&0\end{bmatrix}$,
	$Z=\begin{bmatrix}1& 0\\0&-1\end{bmatrix}$.\\ \\
	 For any matrices A, through  det$|A-\lambda I|=0$,we can caculate the eigenvalues. \\
	Firstly, we discuss $X$. We set det$|X-\lambda I|=0$, then \\ \\ 
	$\left|\begin{bmatrix}0& 1\\1& 0\end{bmatrix} - \begin{bmatrix}\lambda& 0\\0& \lambda\end{bmatrix}\right|=0$,
	%$\left |-\lambda& 1\\1&-\lambda \right |=0.$\\  \\
	$\left|\begin{array}{cccc} -\lambda& 1\\1&-\lambda \end{array}\right| =0. $ \\  \\
	The solution of $\lambda$  is $1$ or $-1$.\\
	When $\lambda_{1}=1$, calculate 
    	$(X-\lambda_{1} I )|\lambda_{1}\rangle=0$, \\  \\
	$\begin{bmatrix}-1&1\\1&-1\end{bmatrix}\begin{bmatrix}x_{1}\\x_{2}\end{bmatrix}=$
	$\begin{bmatrix}0\\0\end{bmatrix}$. \\  \\ 
	Thus we get the solution $x_{1}=x_{2}$ .\\ 
	We  assume that $x_{1}=1$, then $x_{2}=1$.  \\ 
	After normalization, the following eigenvector is obtained:\\  \\
	$|\lambda_{1}\rangle=\frac{1}{\sqrt{2}}\begin{bmatrix}1\\1\end{bmatrix}
	=\frac{1}{\sqrt{2}}\begin{bmatrix}1\\0\end{bmatrix}+\frac{1}{\sqrt{2}}\begin{bmatrix}0\\1\end{bmatrix}
	=|+\rangle$. \\  \\
	%-1
	When $\lambda_{2}=-1$, calculate
	$(X-\lambda_{2} I)|\lambda_{2}\rangle=0$, \\  \\
	$\begin{bmatrix}-1&1\\1&-1\end{bmatrix}\begin{bmatrix}x_{1}\\x_{2}\end{bmatrix}=$
	$\begin{bmatrix}0\\0\end{bmatrix}$. \\  \\ 
	Thus we get the solution $x_{1}=-x_{2}$.\\ 
	We  assume that $x_{1}=1$, then $x_{2}=-1$.  \\
	After normalization, the following eigenvector is obtained:\\  \\
	$|\lambda_{2}\rangle=\frac{1}{\sqrt{2}}\begin{bmatrix}1\\-1\end{bmatrix}
	=\frac{1}{\sqrt{2}}\begin{bmatrix}1\\0\end{bmatrix}-\frac{1}{\sqrt{2}}\begin{bmatrix}0\\1\end{bmatrix}
	=|-\rangle$.
	\\ \\
	Thus the diagonal representation of $X$ is$|+\rangle\langle+|-|-\rangle\langle-|.$ \\ 
	Secondly, we discuss $Y$. We set  det$|Y-\lambda I|=0$, then\\  \\
	$\left| \begin{bmatrix}1&-i\\i& 0\end{bmatrix}-\begin{bmatrix}\lambda&0\\0&\lambda\end{bmatrix} \right |=0$,
	%$\begin{bmatrix}-\lambda&-i\\i&-\lambda\end{bmatrix}$=0.\\  \\
	$\left|\begin{array}{cccc} -\lambda&-i\\i&-\lambda \end{array}\right| =0. $ \\

	The solution of $\lambda$  is $1$ or $-1$.
	
	When $\lambda_{1}=1$, calculate
    	$(Y-\lambda_{1} I)|\lambda_{1}\rangle=0$, \\  \\
	$\begin{bmatrix}-1&-i\\i&-1\end{bmatrix}\begin{bmatrix}x_{1}\\x_{2}\end{bmatrix}$=
	$\begin{bmatrix}0\\0\end{bmatrix}$. \\  \\ 
	So we get the solution $x_{1}=-ix_{2}$ .\\
	We  assume that $x_{1}=1$, then $x_{2}=i$.  \\ 
	%归一化Y
	After normalization, the following eigenvector is obtained:\\  \\
	$|\lambda_{1}\rangle=\frac{1}{\sqrt{2}}\begin{bmatrix}1\\i\end{bmatrix}.$ 
	\\  \\
	When $\lambda_{2}=-1$, calculate
    	$(Y-\lambda_{2} I)|\lambda_{2}\rangle=0,$ \\  \\
	$\begin{bmatrix}1&-i\\i&1\end{bmatrix}\begin{bmatrix}x_{1}\\x_{2}\end{bmatrix}=$
	$\begin{bmatrix}0\\0\end{bmatrix}$. \\  \\
	So we get the solution $x_{1}=ix_{2}$.
	We  assume that $x_{1}=1$, thus $x_{2}=-i$.\\
	%归一化Y
	After normalization, the following eigenvector is obtained:\\  \\
	$|\lambda_{2}\rangle=\frac{1}{\sqrt{2}}\begin{bmatrix}1\\-i\end{bmatrix}.$ \\  \\
	%对角化
	So we can get $Y=\left[\begin{array}{c}{\frac{\sqrt{2}}{2}} \\ {\frac{\sqrt{2}}{2} i}\end{array}\right]\left[\begin{array}{cc}{\frac{\sqrt{2}}{2}} & {\frac{\sqrt{2}}{2} i}\end{array}\right]-\left[\begin{array}{c}{\frac{\sqrt{2}}{2}} \\ {-\frac{\sqrt{2}}{2} i}\end{array}\right]\left[\begin{array}{cc}{\frac{\sqrt{2}}{2}} & {-\frac{\sqrt{2}}{2} i}\end{array}\right]\\ \\=\frac{1}{2}\left[\left[\begin{array}{c}{1} \\ {i}\end{array}\right]\left[\begin{array}{cc}{1} & {i}\end{array}\right]-\left[\begin{array}{c}{1} \\ {-i}\end{array}\right]\left[\begin{array}{cc}{1} & {-i}\end{array}\right]\right].$ \\  \\
	Thirdly, we discuss $Z$. We set det$|Z-\lambda I |=0$, then\\  \\
	$\left |\begin{bmatrix}1&0\\0&-1\end{bmatrix}-\begin{bmatrix}\lambda&0\\0&\lambda\end{bmatrix}\right|=0,$
	%$\begin{bmatrix}1-\lambda&0\\0&-1-\lambda\end{bmatrix}=0$,\\  \\
	$\left|\begin{array}{cccc}1-\lambda&0\\0&-1-\lambda \end{array}\right| =0. $ \\ \\

	The solution of $\lambda$ is $1$ or $-1$.
	
	When $\lambda_{1}=1$, calculate
    	$(Z-\lambda_{1} I)|\lambda_{1}\rangle=0,$ \\  \\ 
	$\begin{bmatrix}0&0\\0&-2\end{bmatrix}\begin{bmatrix}x_{1}\\x_{2}\end{bmatrix}=$
	$\begin{bmatrix}0\\0\end{bmatrix}$. \\  
	
	The solution is $x_{2}=0$. \\
	We  assume that $x_{1}=1$,
	thus $\ket{\lambda_{1}}=\begin{bmatrix}1\\0\end{bmatrix}$. 
	\\
	When $\lambda_{2}=-1$, calculate
    	$(Z-\lambda_{2} I )|\lambda_{2}\rangle=0,$ \\  \\
	$\begin{bmatrix}2&0\\0&0\end{bmatrix}\begin{bmatrix}x_{1}\\x_{2}\end{bmatrix}=$
	$\begin{bmatrix}0\\0\end{bmatrix}$. \\  \\ 
	The solution is $x_{1}=0$.\\
	We  assume that $x_{2}=1$,  \\ 
	\\
	thus $\ket{\lambda_{2}}=\begin{bmatrix}0\\1\end{bmatrix},
	 Z=\begin{bmatrix}1&0\\0&-1\end{bmatrix} 
	=\begin{bmatrix}1&0\\0&0\end{bmatrix}-\begin{bmatrix}0&0\\0&\-1\end{bmatrix}.$ \\ \\
	So the diagonal representation of $Z$ is $|0\rangle\langle 0 |-|1\rangle\langle 1|.$ \\  	
	
\textbf{Exercise 2.12:  } Prove that the matrix   $\begin{bmatrix}1&0\\1	&1\end{bmatrix}$ is not diagonalizable. 
	\\ 
\textbf{Answer:}\\
	The necessary and sufficient condition for diagonalization is that there are n linearly independent eigenvectors for n-order square matrices. According to the knowledge of linear algebra elementary transformation,\\ \\
	 $\begin{bmatrix}1&0\\1&1\end{bmatrix}\xrightarrow{second\,\, line \,\,minus\,\, first\,\, line} \begin{bmatrix}1&0\\0&1\end{bmatrix}$. They have same eigenvalue $\lambda=1.$\\Then we compute the eigenvectors,
	$\begin{bmatrix}0&0\\1&0\end{bmatrix}\begin{bmatrix}x_{1}\\x_{2}\end{bmatrix}
	=\begin{bmatrix}0\\0\end{bmatrix}$.\\
	Thus we get the solution is $x_{1}=0$, and we set $x_{2}=1$. So the eigenvector is $k\begin{bmatrix}0\\1\end{bmatrix}$,  $k\neq0$. \\ \\
	Only have one eigenvalue and one anti-linearly dependent eigenvector of the matrix does not satisfy the necessary and sufficient conditions, so the matrix is not diagonalizable. 
	\\
	
\textbf{Exercise 2.13:} If $|w\rangle$ and $|v\rangle$ are any two vectors, show that $(|w\rangle\langle v|)^{\dagger}=|v\rangle\langle w|$. \\
\textbf{Answer:}\\
	Suppose $A ,B$ is any linear operator on a Hilbert space $V$. For all vectors $ |w\rangle ,| v\rangle \in V$, \\ 
	since $(AB)^{\dagger}=B^{\dagger}A^{\dagger}$ and  $|v\rangle^{\dagger}\equiv\langle v|$, we can get $(|v\rangle\langle w|)^{\dagger}=\langle w|^{\dagger} | v\rangle^{\dagger}=|w\rangle\langle v|$. 
	\\ \\
\textbf{Exercise 2.14:(Anti-linearity of the adjoint)} Show that the adjoint operation is anti-linear,
	 $\left(\sum_{i} a_{i} A_{i}\right)^{\dagger}=\sum_{i} a_{i}^{*} A_{i}^{\dagger}$. \\  
\textbf{Answer:}\\
	 	Suppose $A_{i}$ is any linear operator on a Hilbert space $V$ and we set all vectors $ |w\rangle ,| v\rangle \in V$. 
According to  inner product is linear in the second argument and  conjugate-linear in the first argument, we can make the 	following derivation: 

	$\left(|v\rangle, \sum_{i} a_{i}^{*} A_{i}^{\dagger}\left|w_{i}\right\rangle\right)$ 
	$= \sum_{i} a_{i}^{*} \left(|v\rangle,A_{i}^{\dagger}\left|w_{i}\right\rangle\right)$  \\
	$=\sum_{i} a_{i}^{*} \left(A_{i}|v\rangle,\left|w_{i}\right\rangle\right)$
	$=\left(\sum_{i} a_{i}A_{i}|v\rangle,\left|w_{i}\right\rangle\right)$
	$=\left(|v\rangle, (\sum_{i} a_{i} A_{i})^{\dagger}\left|w_{i}\right\rangle\right)$.\\
	Thus the adjoint operation is anti-linear, $(\sum_{i} a_{i} A_{i})^{\dagger}=\sum_{i} a_{i}^{*} A_{i}^{\dagger}$.  \\ 
	
\textbf{Exercise 2.15:} Show that $\left(A^{\dagger}\right)^{\dagger}=A.$  \\
	\textbf{Answer:}\\
	Suppose $A$ is any linear operator on a Hilbert space $V$ and we set  $ |w_{i}\rangle ,| v\rangle \in V$. \\ 
Since $\left(|v\rangle, A^{\dagger}|w\rangle\right)=\left(A^{\dagger}|w\rangle,|v\rangle\right)^{*}=\left(|w\rangle, A|v\rangle\right)^{*}=\left(A|v\rangle,|w\rangle\right)$
	and $\left(|v\rangle, A^{\dagger}|w\rangle\right)=\left[\left(A^{\dagger}\right)^{\dagger}|v\rangle,|w\rangle\right].$
	\\
	Thus we proved that $\left(A^{\dagger}\right)^{\dagger}=A$.
	\\  \\ 
\textbf{Exercise 2.16:} Show that any projector $P$ satisfies the equation $P^{2}=P$. \\
\textbf{Answer:}\\
	Suppose $V$ is a  Hermite space, $W$ is the k-dimensional subspace of d-dimensional vector space $V$. Using the Gram-Schimdt process, we can construct $ \ket{1}, \ket{2}, . . .\ket{d}$ as a set of standard orthogonal basis of $V$, so that $ \ket{1}, \ket{2}, . . .\ket{k} $is a standard orthogonal basis of $W  ,P \equiv \sum_{i=1}^{k}|i\rangle\langle i|.$ \\ \\
	The proof process is as follows: \\
	$\begin{aligned} P^{2} &=(\sum_{i=1}^{k}|i\rangle\langle i|)(\sum_{j=1}^{k}|j\rangle\langle j|)=\sum_{i=1}^{k}\sum_{j=1}^{k}|i\rangle\langle i | j\rangle\langle j| \\ 	&=\sum_{i=1}^{k}\sum_{j=1}^{k}|i\rangle \delta_{i j}\langle j|=\sum_{i=1}^{k}| i\rangle\langle i|=P. \end{aligned}$\\  \\  
	\\
\textbf{Exercise 2.17:} Show that a normal matrix is Hermitian if and only if it has real eigenvalues.
	\\ 
	\textbf{Answer:}\\
	Suppose A is a normal matrix, so it can be given a spectral decomposition,  $A= \sum_{i}\lambda_{i}|i\rangle\langle i|$ and  $A^{\dagger} = \sum_{i}\lambda_{i}^{*}|i\rangle\langle i|, (\lambda_{i}\geq 0)$. \\
	Since $A$ is a Hermitian operators, we have $A=A^{\dagger}$, then $\sum_{i}\lambda_{i}|i\rangle\langle i|= \sum_{i}\lambda_{i}^{*}|i\rangle\langle i|$. Thus we can get $\lambda_{i} =\lambda_{i}^{*}$, then $\lambda_{i} \in R$.
	Since $\lambda_{i} \in R$, we can get $\lambda_{i} =\lambda_{i}^{*}$ and $\sum_{i}\lambda_{i}|i\rangle\langle i|= \sum_{i}\lambda_{i}^{*}|i\rangle\langle i|$. Thus we can get $A=A^{\dagger}$. $A$ is  Hermitian. \\
	Above all, we proved that a normal matrix is Hermitian if and only if it has real eigenvalues.
	\\ \\
\textbf{Exercise 2.18:} Show that all eigenvalues of a unitary matrix have modulus $1$, that is, can be written in the form $e^{i\theta}$ for some real ${\theta}$. \\
\textbf{Answer:}\\

	 Suppose $U$ is a unitary matrix, thus $U$ is normal and can be given a spectral decomposition, $U= \sum_{i}\lambda_{i}|i\rangle\langle i|, (\lambda_{i}\geq 0)$.  The derivation process is as follows: \\
	$U$ satisfies $UU^{\dagger}=I(I \equiv \sum_{i}|i\rangle\langle i|).$ \\
	${U U^{\dagger}=\left(\sum_{i} \lambda_{i}|i\rangle \langle i|\right)\left(\sum_{i} \lambda_{i}|i\rangle\langle i|\right)^{\dagger}=\sum_{i} \lambda_{i} \lambda_{i}^{*}|i\rangle\langle i|=I}. \\ {\sum_{i} \lambda_{i} \lambda_{i}^{*}|i\rangle\langle i|=\sum_{i}| i\rangle\langle i| \Rightarrow \forall i, \lambda_{i} \lambda_{i}^{*}=1}$.
	\\
	Since $\lambda_{i} \lambda_{i}^{*}=1$,  we can get $\left\|\lambda_{i}\right\|=1$. \\ 
	Let $\lambda_{i}={e}^{i \theta}=\cos \theta+{i} \sin \theta$, then $ \lambda_{i}^{*}={e}^{-i \theta}=\cos \theta-{i} \sin \theta.$
	\\
	Since ${e}^{i \theta}*({e}^{i \theta})^{*}=(\cos \theta+{i} \sin \theta)*(\cos \theta-{i} \sin \theta)=1$, $\lambda_{i}$ can be written in the form ${e}^{i \theta}$ for some real $\theta$.
	\\
	\\  
\textbf{Exercise 2.19: (Pauli matrices: Hermitian and unitary)} Show that the Pauli matrices are Hermitian and unitary. 
\\
\textbf{Answer:}\\
	For $Y$ is an example.\\
	Hermitian:\\
	$Y^{\dagger}=\left(Y^{*}\right)^{\mathrm{T}}=\left(\left[\begin{array}{cc}{0} & {-i} \\ {i} & {0}\end{array}\right]^{\mathrm{*}}	\right)^{\mathrm{T}}=\left[\begin{array}{cc}{0} & {-i} \\ {i} & {0}\end{array}\right]=Y.$ 
	\\ \\
	 Unitary:\\
	$\begin{aligned} Y^{\dagger} Y &=\left(Y^{*}\right)^{T} Y=\left(\left[\begin{array}{cc}{0} & {-i} \\ {i} & {0}\end{array}	\right]^{*}\right)^{T}\left[\begin{array}{cc}{0} & {-i} \\ {i} & {0}\end{array}\right]=\left[\begin{array}{cc}{0} & {i} \\ {-i} & {0}\end{array}\right]^{T}\left[\begin{array}{cc}{0} & {-i} \\ {i} & {0}\end{array}\right] \\ &=\left[\begin{array}{cc}{0} & {-i} \\ {i} & 	{0}\end{array}\right]\left[\begin{array}{cc}{0} & {-i} \\ {i} & {0}\end{array}\right]=\left[\begin{array}{cc}{1} & {0} \\ {0} & 	{1}\end{array}\right]=I. \end{aligned}$\\ 
	\\ \\
	Thus $ Y$ is Hermitian and unitary. 
	\\
	Other cases will also reach corresponding conclusions according to the above calculations. 
	\\ \\
\textbf{Exercise 2.20: (Basis changes)} Suppose $A^{\prime}$ and  $A^{\prime\prime}$ are matrix representations of an
operator $A$ on a vector space $V$ with respect to two different orthonormal bases,$| v_{i}\rangle$ and $| w_{j}\rangle$.Then the elements of $A^{\prime}$  and $A^{\prime\prime}$ are $A_{i j}^{\prime}=\langle v_{i}|A| v_{j}\rangle$ and $A_{i j}^{\prime \prime}=\langle w_{i}|A| w_{j}\rangle$. Characterize the relationship between $A^{\prime}$ and  $A^{\prime\prime}$. \\
\textbf{Answer:}\\
Suppose  $U \equiv \sum_{i}\left|w_{i}\right\rangle\left\langle v_{i}\right|, U$ is  unitary operator, then we can make the following derivation: \\
	$\begin{aligned} A_{i j}^{\prime} &=\left\langle v_{i}|A| v_{j}\right\rangle \\ &=\left\langle v_{i}\left|U U^{\dagger} A UU^{\dagger}\right| v_{j}\right\rangle \\ &=\sum_{p, q, r, s}\left\langle v_{i} | w_{p}\right\rangle\left\langle v_{p} | v_{q}\right\rangle\left\langle w_{q}|A| w_{r}\right\rangle\left\langle v_{r} | v_{s}\right\rangle\left\langle w_{s} | v_{j}\right\rangle \\ &=\sum_{p, q, r, s}\left\langle v_{i} | w_{p}\right\rangle \delta_{p q} A_{q r}^{\prime \prime} \delta_{r s}\left\langle w_{s} | v_{j}\right\rangle \\ &=\sum_{p, r}\left\langle v_{i} | w_{p}\right\rangle\left\langle w_{r} | v_{j}\right\rangle A_{p r}^{\prime \prime}. \end{aligned}$ 
	\\
	Suppose P=$\sum_{i j}p_{i j}$ is a row elementary matrix and its elements are $p_{i j}= \bra{w_j}\ket{v_{i}}$, we can get  $A_{ij}^{\prime}=\sum_{p, r}p_{ip}^{*}A_{p r}^{\prime \prime}p_{jr}$.  Thus there is a row elementary matrix $P$ between $A$ and $B$ such that $A^{\prime}=P^{\dagger} A^{\prime\prime} P$ holds.
	
	
\end{quote}

\end{document}