\documentclass[UTF8]{ctexart}
\usepackage{physics}
\usepackage{amsmath}
\usepackage{geometry}
\usepackage{indentfirst} 
 %\mathinner{\langle a | }\quad \mathinner{ | b \rangle}
% \def\bra#1{\mathinner{\langle{#1}}


 \setlength{\parindent}{2em}
\geometry{a4paper,scale=0.8}
\begin{document}
	\title{\textbf{Q\&A(2.31-2.40)}\\[1ex]\begin{large}
		\end{large}}
	\author{LuoTingyu\quad JiangHui}
	\maketitle
\begin{quote}
\textbf{Exercise 2.41: (Anti-commutation relations for the Pauli matrices)} Verify the
anti-commutation relations
\begin{center}
	$\{\sigma_{i},\sigma_{j}\} = 0$
\end{center}
where $i \neq j$ are both chosen from the set 1,2,3. Also verify that $(i = 0,1,2,3)$
\begin{center}
	$\sigma_{i}^{2} = I$.
\end{center}
\textbf{Answer:}\\
$\sigma_{0}=
\begin{bmatrix}
1 & 0 \\
0 & 1	
\end{bmatrix}$,
$\sigma_{1}=
\begin{bmatrix}
0 & 1 \\
1 & 0	
\end{bmatrix}$,
$\sigma_{2}=
\begin{bmatrix}
0 & -i \\
i & 0	
\end{bmatrix}$,
$\sigma_{3}=
\begin{bmatrix}
1 & 0 \\
0 & -1	
\end{bmatrix}$.
\\
When $i=1, j=2$,we can get the follow equation,
\\
\begin{equation}
	\begin{aligned}
		\{\sigma_{1},\sigma_{2}\}=&
			\begin{bmatrix}
				0 & 1 \\
				1 & 0	
				\end{bmatrix}
				\begin{bmatrix}
					0 & -i \\
					i & 0	
					\end{bmatrix}
				+
				\begin{bmatrix}
					0 & -i \\
					i & 0	
					\end{bmatrix}
				\begin{bmatrix}
					0 & 1 \\
					1 & 0	
					\end{bmatrix}
					\\
					=&\begin{bmatrix}
						i & 0 \\
						0 & -i	
						\end{bmatrix}
						+
						\begin{bmatrix}
							-i & 0 \\
							0 & i	
							\end{bmatrix}
							\\
					=&0.
	\end{aligned}
	\end{equation}
	Similarly avaliable, we verify that $\{\sigma_{i},\sigma_{j}\} = 0$.
\\
When $i=1, j=2$, we can get the follow equation,
\begin{equation}
	\begin{aligned}
		\sigma_{0}^{2}=&
			\begin{bmatrix}
				0 & 1 \\
				1 & 0	
				\end{bmatrix}
				\begin{bmatrix}
					0 & 1 \\
					1 & 0	
					\end{bmatrix}
					\\
					=&\begin{bmatrix}
						1 & 0 \\
						0 & 1	
						\end{bmatrix}
					\\
					=&I.
	\end{aligned}
	\end{equation}
	Similarly avaliable, we verify that $\sigma_{i}^{2} = I$.

\textbf{Exercise 2.42:  } Verify that
\begin{center}
	$AB=\frac{[A,B] + \{A,B\} }{2}$.
\end{center}
\textbf{Answer:}\\
\begin{equation}
	\begin{aligned}
\frac{[A,B] + \{A,B\} }{2}=&\frac{AB-BA + AB+BA }{2}\\
						  =&AB.
	\end{aligned}
	\end{equation}


\textbf{Exercise 2.43:} Show that for $j,k = 1,2,3,$
\begin{center}
$\sigma_{j}\sigma_{k}=\delta_{jk}I+i\sum_{l=1}^{3}\epsilon_{jkl}\sigma_{l}.$
\end{center}

\textbf{Answer:}\\

$\sigma_{j}\sigma_{j}=\delta_{jj}I+i\sum_{l=1}^{3}\epsilon_{jjl}\sigma_{l}$, according to the exercise 2.40, 
we can know the equations $\epsilon_{jjl}=0 (l=1,2,3,j=1,2,3)$ and $\sigma_{j}\sigma_{j}=\sigma_{j}^{2}=I$. \\
Thus when $j=k$,
$\sigma_{j}\sigma_{k}=\delta_{jk}I+i\sum_{l=1}^{3}\epsilon_{jkl}\sigma_{l}.$\\

When $j=1,k=2$,we can get the follow equations,\\
$\sigma_{1}\sigma_{2}=0+i\epsilon_{121}\sigma_{1}+i\epsilon_{122}\sigma_{2}+i\epsilon_{123}\sigma_{3}
=\begin{bmatrix}
	i & 0 \\
	0 & -i	
	\end{bmatrix}$,\\
$
			\sigma_{1}\sigma_{2}=
				\begin{bmatrix}
					0 & 1 \\
					1 & 0	
					\end{bmatrix}
					\begin{bmatrix}
						0 & -i \\
						i & 0	
						\end{bmatrix}
						=\begin{bmatrix}
							i & 0 \\
							0 & -i	
							\end{bmatrix}
$.\\
	Thus $\sigma_{1}\sigma_{2}=\sigma_{12}I+i\sum_{l=1}^{3}\epsilon_{12l}\sigma_{l}$.
\\
	When $j=2,k=1$,we can get the follow equations,\\
$\sigma_{2}\sigma_{1}=0+i\epsilon_{211}\sigma_{1}+i\epsilon_{212}\sigma_{2}+i\epsilon_{213}\sigma_{3}
=\begin{bmatrix}
	-i & 0 \\
	0 & i	
	\end{bmatrix}$,\\
$
			\sigma_{2}\sigma_{1}=
					\begin{bmatrix}
						0 & -i \\
						i & 0	
						\end{bmatrix}
				\begin{bmatrix}
					0 & 1 \\
					1 & 0	
					\end{bmatrix}
						=\begin{bmatrix}
							-i & 0 \\
							0 & i	
							\end{bmatrix}.
$\\
	Thus $\sigma_{2}\sigma_{1}=\sigma_{21}I+i\sum_{l=1}^{3}\epsilon_{21l}\sigma_{l}$.
	\\
	Similarly, we can get the equations: \\
	$\sigma_{1}\sigma_{3}=\sigma_{13}I+i\sum_{l=1}^{3}\epsilon_{13l}\sigma_{l}$\\
	$\sigma_{3}\sigma_{1}=\sigma_{31}I+i\sum_{l=1}^{3}\epsilon_{31l}\sigma_{l}$\\
	$\sigma_{2}\sigma_{3}=\sigma_{23}I+i\sum_{l=2}^{3}\epsilon_{23l}\sigma_{l}$\\
	$\sigma_{3}\sigma_{2}=\sigma_{3}I+i\sum_{l=2}^{3}\epsilon_{32l}\sigma_{l}$.\\
	In summary, we proved the $\sigma_{j}\sigma_{k}=\delta_{jk}I+i\sum_{l=1}^{3}\epsilon_{jkl}\sigma_{l}.$
for	$j,k,l = 1,2,3,$.
\\
		\textbf{Exercise 2.44: } Suppose $[A, B] = 0, \{A, B\} = 0,$ and $A$ is invertible.
 Show that $B$ must be 0s. \\
\textbf{Answer:}\\
We can get the follow equations: \\
$[A, B] = 0  \rightarrow AB-BA=0  $         (1)\\
$\{A, B\} = 0 \rightarrow AB+BA=0 $           (2).\\
Add up the above equations, the solution is 
$AB=0$. \\
Since $A$ is invertible,$A$ can't be zero matrix, then $B$ must be 0s.
\\
\\
\textbf{Exercise 2.45: } 
Show that $[A, B]^{\dagger} = [B^{\dagger}, A^{\dagger}].$
\\
\textbf{Answer:}\\
$[A, B]^{\dagger}=(AB-BA)^{\dagger}=(AB)^{\dagger}-(BA)^{\dagger}=B^{\dagger}A^{\dagger}-A^{\dagger}B^{\dagger}
=[B^{\dagger}, A^{\dagger}]$.
\\
\\
\textbf{Exercise 2.46: } Show that $[A, B] = −[B, A]$.
\\
\textbf{Answer:}\\
$[A, B] =AB-BA =-(BA-AB) =-[B,A]$.
\\
\\
\textbf{Exercise 2.47: } Suppose $A$ and $B$ are Hermitian. Show that $i[A, B]$ is Hermitian.
\\
	\textbf{Answer:}\\
If we want to prove that $i[A, B]$ is Hermitian, we can prove that $(i[A, B])^{\dagger}=i[A, B]$.
 Konwn that $A$ and $B$ are Hermitian, according to exercise 2.45 and exercise 2.46
 we can do the following derivation: \\
$(i[A, B])^{\dagger}= -i[B^{\dagger}, A^{\dagger}]=-i[B,A]=i[A,B]$.\\
\\
\\
\textbf{Exercise 2.48: } What is the polar decomposition of a positive matrix $P$ ? 
Of a unitary matrix $U$?
Of a Hermitian matrix, $H$?
 \\
\textbf{Answer:}\\
Since $P$ is a positive matrix and it is diagonalizable. Then $P=\sum_{i}\lambda_{i}\ket{i}\bra{i},\lambda_{i}\geq0. $\\
$J=\sqrt{P^{\dagger}P}=\sqrt{P^{2}} =\sum_{i}\lambda_{i}^{2}\ket{i}\bra{i}=P$.\\
Therefore polar decomposition of $P$ is $P = UP$ for all $P$. Thus $U = I$, then $P = P.$
\\
Since $U$ is a unitary matrix,then $U$ can be decomposed by $U=WJ$ where $W$ is unitary and $J$ is positive, $J=\sqrt{U^{\dagger}U}$.
$J=\sqrt{U^{\dagger}U}=\sqrt{I}=I.$\\
Since unitary operators are invertible, $W = UJ^{−1} = UI^{−1} = UI = U$. Thus polar decomposition of $U$ is $U = U$.
\\
Suppose $H = UJ.$$J=\sqrt{H^{\dagger}H}=\sqrt{H^{2}}$.\\
For spectral decomposition, $H=\sum_{i}\lambda_{i}\ket{i}\bra{i},\lambda_{i}\in R$.\\
$\sqrt{H^{\dagger}H}=\sum_{i}\sqrt{\lambda_{i}^{2}}\ket{i}\bra{i}=\sum_{i}|\lambda_{i}|\ket{i}\bra{i}\neq H$.\\
Thus $H=U\sqrt{H^{2}}$.
\\
\\
\textbf{Exercise 2.49: } Express the polar decomposition of a normal matrix in the outer product representation.
\\
\textbf{Answer:}\\

Suppose $A$ is a normal matrix, then $A$ is diagonalizable, $A=\sum_{i}\lambda_{i}\ket{i}\bra{i}$.\\

\begin{equation}
	\begin{aligned}
J=&\sqrt{A^{\dagger}A}=\sum_{i}|\lambda_{i}|\ket{i}\bra{i}\\
U=&\sum_{i}\ket{e_{i}}\bra{i}\\
A=&UJ\\
 =&\sum_{i}\ket{e_{i}}\bra{i}*\sum_{i}|\lambda_{i}|\ket{i}\bra{i}\\
 =&\sum_{i}|\lambda_{i}|\ket{e_{i}}\bra{i}
	\end{aligned}
	\end{equation}
\\
\\
\textbf{Exercise 2.50: }  Find the left and right polar decompositions of the matrix
\begin{center}
$\begin{bmatrix}1& 0\\1& 1\end{bmatrix}$.
\end{center}

\textbf{Answer:}	 \\
\begin{equation}
	\begin{aligned}
J=&\sqrt{A^{\dagger}A}=\sqrt{\begin{bmatrix}2& 1\\1& 1\end{bmatrix}}\\
U=&\sum_{i}\ket{e_{i}}\bra{i}\\
A=&UJ=\sum_{i}\ket{e_{i}}\bra{i}*\sum_{i}|\lambda_{i}|\ket{i}\bra{i}\\
 =&\sum_{i}|\lambda_{i}|\ket{e_{i}}\bra{i}
	\end{aligned}
	\end{equation}



\end{quote}

\end{document}