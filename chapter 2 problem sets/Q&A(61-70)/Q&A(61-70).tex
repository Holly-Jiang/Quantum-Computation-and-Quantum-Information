\documentclass[UTF8]{ctexart}
\usepackage{physics}
\usepackage{amsmath}
\usepackage{geometry}
\usepackage{indentfirst} 
 %\mathinner{\langle a | }\quad \mathinner{ | b \rangle}
% \def\bra#1{\mathinner{\langle{#1}}


 \setlength{\parindent}{2em}
\geometry{a4paper,scale=0.8}
\begin{document}
	\title{\textbf{Q\&A(2.61-2.70)}\\[1ex]\begin{large}
		\end{large}}
	\author{LuoTingyu\quad JiangHui}
	\maketitle
\begin{quote}
\textbf{Exercise 2.61: } Calculate the probability of obtaining the result 
$+1$ for a measurement $\vec{v}\cdot \vec{\sigma}$, given that the state 
prior to measurement is $\ket{0}$. What is the state of the system after 
the measurement if $+1$ is obtained?
\\
\textbf{Answer:}\\
\begin{equation}
	\begin{aligned}
		p(+1)=&\bra{0}\ket{\lambda_{1}}\bra{\lambda_{1}}\ket{0} \\
			 =&\frac{1}{2}\bra{0}(I+\vec{v}\cdot\vec{\sigma})\ket{0} \\
			 =&\frac{1}{2}\begin{bmatrix}
				 1 & 0
			 \end{bmatrix}\begin{bmatrix}
				1+v_{3} & v_{1}-iv_{2} \\ v_{1}+iv_{2} & 1-v_{3}
			\end{bmatrix}
			\begin{bmatrix}
				1 \\ 0
			\end{bmatrix}
			\\
			=&\frac{1}{2}(1+v_{3}).
	\end{aligned}
\end{equation} 
The state of the quantum system immediately after the measurement is
\\
\begin{equation}
	\begin{aligned}
		\frac{\ket{\lambda_{1}}\bra{\lambda_{1}}\ket{0}}{\sqrt{p(+1)}}=&
		\frac{\ket{\lambda_{1}}\bra{\lambda_{1}}\ket{0}}{\sqrt{\frac{1}{2}(1+v_{3})}}\\
		=&\frac{1}{\sqrt{\frac{1}{2}(1+v_{3})}}*\frac{1}{2}\begin{bmatrix}
			1+v_{3} & v_{1}-iv_{2} \\ v_{1}+iv_{2} & 1-v_{3}
		\end{bmatrix}
		\begin{bmatrix}
			1 \\ 0
		\end{bmatrix}\\
		=&\frac{1}{\sqrt{\frac{1}{2}(1+v_{3})}}\frac{1}{2}
		\begin{bmatrix}
			1+v_{3} \\v_{1}+iv_{2}
		\end{bmatrix} \\
		=&\sqrt{\frac{1+v_{3}}{2}}\begin{bmatrix}
			1 \\ \frac{v_{1}+iv_{2}}{1+v_{3}}
		\end{bmatrix}.
	\end{aligned}
\end{equation}
\\
\\
\textbf{Exercise 2.62:  } Show that any measurement where the measurement operators and the
POVM elements coincide is a projective measurement.
\\
\textbf{Answer:}\\
Suppose $M_{m}$ is a measurement operator, From the assumption, 
$E_{m}=M_{m}^{\dagger}M_{m}=M_{m}$.\\
\begin{equation}
	\begin{aligned}
		p(m)=&\ket{\psi}E_{m}\bra{\psi}\\
			=&\ket{\psi}M_{m}\bra{\psi} \\
			\geq&0 \\
			=&(\ket{\psi},M_{m}\ket{\psi})
	\end{aligned}
\end{equation}
Thus, $M_{m}$ is an positive operator for all $\ket{\psi}$, then $M_{m}$
 is Hermitian, $M_{m}^{\dagger}=M_{m}$.\\
 \begin{equation}
	\begin{aligned}
		E_{m}=M_{m}^{\dagger}M_{m}=M_{m}^{2}=M_{m}	
	\end{aligned}
\end{equation}
Thus, any measurement where the measurement operators and the
POVM elements coincide is a projective measurement.
\\
\\
\textbf{Exercise 2.63:} Suppose a measurement is described by measurement operators $M_{m}$.
Show that there exist unitary operators $U_{m}$ such that $M_{m} = U_{m}sqrt{E_{m}}$, where
   $E_{m}$ is the POVM associated to the measurement.
   \\
\textbf{Answer:}\\
Since $E_{m}$ is a positive operator.
\begin{equation}
	\begin{aligned}
		M_{m}^{\dagger}M_{m}=&(U_{m}sqrt{E_{m}})^{\dagger}U_{m}sqrt{E_{m}}\\
			 				=&\sqrt{E_{m}}^{\dagger}U_{m}^{\dagger}U_{m}sqrt{E_{m}}\\
							=&\sqrt{E_{m}}^{\dagger}\sqrt{E_{m}}\\
							=&E_{m}
	\end{aligned}
\end{equation}
Since  $E_{m}$ is POVM, for arbitrary unitary $U$, $M_{m}^{\dagger}M_{m}$ is POVM.
\\
\\
\textbf{Exercise 2.64: } Suppose Bob is given a quantum state chosen from a set $\ket{\psi_{1}}, . . . , \ket{\psi_{m}}$
 of linearly independent states. Construct a POVM $\{E_{1},E_{2},...,E_{m+1}\}$ such that if outcome $E_{i}$ occurs,
  $1 \leq i \leq m$, then Bob knows with certainty that he was given the state $\ket{\psi_{i}}$. 
  (The POVM must be such that $\bra{\psi_{i}}E_{i}\ket{\psi_{i}}>0$ for each $i$.)
\textbf{Answer:}\\
	\\
	\\
\textbf{Exercise 2.65: } 
Express the states $\frac{(\ket{0}+\ket{1})}{2}$ and  $\frac{(\ket{0}-\ket{1})}{2}$ in a basis in
which they are not the same up to a relative phase shift.
\\
\textbf{Answer:}\\
\begin{equation}
	\begin{aligned}
		\ket{+}=&\frac{(\ket{0}+\ket{1})}{2} \,\,\,
		\ket{-}=&\frac{(\ket{0}-\ket{1})}{2}
	\end{aligned}
	\end{equation}
\\
\\
\textbf{Exercise 2.66: } Show that the average value of the observable $X_{1}Z_{2}$ for a two qubit
system measured in the state $\frac{(\ket{00}+\ket{11})}{\sqrt{2}}$ is zero.\\
\textbf{Answer:}\\
The $X$ matrix takes $\ket{0}$ to $\ket{1}$, and $\ket{1}$ to $\ket{0}$, and the $Z$ matrix leaves $\ket{0}$ invariant, and takes $\ket{1}$ to $-\ket{1}$.
\begin{equation}
	\begin{aligned}
		E(m)=&\frac{(\bra{00}+\bra{11})}{\sqrt{2}}X_{1}Z_{2}\frac{(\ket{00}+\ket{11})}{\sqrt{2}}\\
			=&\frac{(\bra{00}+\bra{11})}{\sqrt{2}}\frac{(\ket{10}-\ket{01})}{\sqrt{2}}\\
			=&\frac{1}{2}\begin{bmatrix}
				1 & 1 \\ 1 & 1
			\end{bmatrix}
			\begin{bmatrix}
				-1 & 1 \\ 1 & -1
			\end{bmatrix} \\
			=&0
\end{aligned}
\end{equation}
\\
\\
\textbf{Exercise 2.67: } 
Suppose $V$ is a Hilbert space with a subspace $W$ . Suppose
$U:W \rightarrow V$ is a linear operator which preserves inner products, 
that is, for any $\ket{w_{1}}$ and $\ket{w_{2}}$ in $W$,
\begin{equation}
	\begin{aligned}
 \bra{w_{1}}U^{\dagger}U\ket{w_{2}}=&\bra{w_{1}}\ket{w_{2}}
	\end{aligned}
\end{equation}
Prove that there exists a unitary operator $U^{'}: V \rightarrow V$ which $extends$ $U$ .
 That is, $ U^{'}\ket{w} = U\ket{w}$ for all $\ket{w}$ in $W$, but $U^{'}$ is defined on the entire space $V$.
  Usually we omit the prime symbol $′$ and just write $U$ to denote the extension.
\\
	\textbf{Answer:}\\
	\\
\\
\textbf{Exercise 2.68: } Prove that $\ket{\psi}\neq \ket{a}\ket{b}$ for all single qubit states $\ket{a}$ and $\ket{b}$.
 \\
\textbf{Answer:}\\
Suppose $\ket{a}=a_{0}\ket{0}+a_{1}\ket{1}$ and $\ket{b}=b_{0}\ket{0}+b_{1}\ket{1} $.
\begin{equation}
	\begin{aligned}
		\ket{a}\ket{b}=&a_{0}b_{0}\ket{00}+a_{0}b_{1}\ket{01}+a_{1}b_{0}\ket{10}+a_{1}b_{1}\ket{11}\\
	\end{aligned}
\end{equation}
	If $\ket{\psi}= \ket{a}\ket{b}$, then $a_{0}b_{0}=\frac{1}{\sqrt{2}}$, $a_{0}b_{1}=0$, $a_{1}b_{0}=0$, and $a_{1}b_{1}=\frac{1}{\sqrt{2}}$.\\
	If $a_{0}b_{1}=0$, then $a_{0}=0$ or $b_{1}=0$.\\
	When $a_{0}=0$,this does not satisfy $a_{0}b_{0}=\frac{1}{\sqrt{2}}$.
	When $b_{1}=0$,this does not satisfy $a_{1}b_{1}=\frac{1}{\sqrt{2}}$.\\
	If $a_{1}b_{0}=0$, then $a_{1}=0$ or $b_{0}=0$.\\
	When $a_{1}=0$,this does not satisfy $a_{1}b_{1}=\frac{1}{\sqrt{2}}$.
	When $b_{0}=0$,this does not satisfy $a_{0}b_{0}=\frac{1}{\sqrt{2}}$.\\
	Thus $\ket{\psi}\neq \ket{a}\ket{b}$. 
	\\
\\
\textbf{Exercise 2.69: } Verify that the Bell basis forms an orthonormal basis for the two qubit state space.
\\
\textbf{Answer:}\\
Suppose the Bell basis as follows, \\
\begin{equation}
	\begin{aligned}
		\ket{\psi_{1}}=&\frac{\ket{00}+\ket{11}}{\sqrt{2}}=\frac{1}{\sqrt{2}}
		\begin{bmatrix}
			1\\0\\0\\1
		\end{bmatrix}\\
		\ket{\psi_{2}}=&\frac{\ket{00}-\ket{11}}{\sqrt{2}}=\frac{1}{\sqrt{2}}\begin{bmatrix}
			1\\0\\0\\-1
		\end{bmatrix}\\
		\ket{\psi_{3}}=&\frac{\ket{01}+\ket{10}}{\sqrt{2}}=\frac{1}{\sqrt{2}}\begin{bmatrix}
			0\\1\\1\\0
		\end{bmatrix}\\
		\ket{\psi_{4}}=&\frac{\ket{01}-\ket{10}}{\sqrt{2}}=\frac{1}{\sqrt{2}}\begin{bmatrix}
			0\\1\\-1\\0
		\end{bmatrix}\\
	\end{aligned}
\end{equation} 
We need to prove $\{\ket{\psi_{1}}\}$ is linearly independent basis.
\begin{equation}
	\begin{aligned}
		&a_{1}\ket{\psi_{1}}+a_{2}\ket{\psi_{2}}+a_{3}\ket{\psi_{3}}+a_{4}\ket{\psi_{4}}=0\\
		&\begin{bmatrix}
			a_{1}+a_{2}\\
			a_{3}+a_{4}\\
			a_{3}-a_{4}\\
			a_{1}-a_{2}
		\end{bmatrix}=0\\
		&a_{1}=a_{2}=a_{3}=a_{4}=0\\
	\end{aligned}
\end{equation}
Moreover,  $\bra{\psi_{i}}\ket{\psi_{j}}=\delta_{ij} for i,j=1,2,3,4 $ and the norm of $\{\ket{\psi_{1}}\}$ is \\
\begin{equation}
	\begin{aligned}
		\ket{\psi_{1}}=&\sqrt{\bra{\psi_{i}}\ket{\psi_{i}}}\\
					  =&1 \\
	\end{aligned}
\end{equation}
Thus, $\{\ket{\psi_{1}}\}$ forms an orthonormal basis.
\\
\\
\textbf{Exercise 2.70: } Suppose $E$ is any positive operator acting on Alice’s qubit. 
Show that $\bra{\psi}E\otimes I\ket{\psi}$ takes the same value when $\ket{\psi}$ is any of the four Bell states.
 Suppose some malevolent third party (‘Eve’) intercepts Alice’s qubit on the way to Bob 
 in the superdense coding protocol. Can Eve infer anything about which of the four
  possible bit strings 00, 01, 10, 11 Alice is trying to send? If so, how, or if not, why not?
 \\
\textbf{Answer:}	 \\
\begin{equation}
	\begin{aligned}
		\bra{\psi_{1}}E\otimes I\ket{\psi_{1}}=&\frac{1}{2}
		\begin{bmatrix}
			1&0&0&1
		\end{bmatrix}E\otimes I
		\begin{bmatrix}
			1\\0\\0\\1
		\end{bmatrix}
		\\
		=&\frac{1}{2}
		\begin{bmatrix}
			1&0&0&1
		\end{bmatrix}E\otimes
		\begin{bmatrix}
			1\\0\\0\\1
		\end{bmatrix}
		\\
	\end{aligned}
\end{equation}
\end{quote}

\end{document}