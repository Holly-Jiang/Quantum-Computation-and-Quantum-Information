\documentclass[UTF8]{ctexart}
\usepackage{physics}
\usepackage{amsmath}
\usepackage{geometry}
\usepackage{indentfirst} 
 %\mathinner{\langle a | }\quad \mathinner{ | b \rangle}
% \def\bra#1{\mathinner{\langle{#1}}


 \setlength{\parindent}{2em}
\geometry{a4paper,scale=0.8}
\begin{document}
	\title{\textbf{Q\&A(2.51-2.60)}\\[1ex]\begin{large}
		\end{large}}
	\author{LuoTingyu\quad JiangHui}
	\maketitle
\begin{quote}
\textbf{Exercise 2.51: } Verify that the Hadamard gate H is unitary.
\\
\textbf{Answer:}\\
$H^{\dagger}H=\frac{1}{\sqrt{2}}\begin{bmatrix}
	1 & 1 \\1 & -1
\end{bmatrix}
\frac{1}{\sqrt{2}}\begin{bmatrix}
	1 & 1 \\1 & -1
\end{bmatrix}
=\frac{1}{2}\begin{bmatrix}
	2 & 0 \\0 & 2
\end{bmatrix}
=\begin{bmatrix}
	1 & 0 \\0 & 1
\end{bmatrix}
=I$.
\\
Thus the Hadamard gate H is unitary.
\\
\\
\textbf{Exercise 2.52:  } Verify that $H^{2} = I.$
\\
\textbf{Answer:}\\
$H^{2}=\frac{1}{\sqrt{2}}\begin{bmatrix}
	1 & 1 \\1 & -1
\end{bmatrix}
\frac{1}{\sqrt{2}}\begin{bmatrix}
	1 & 1 \\1 & -1
\end{bmatrix}
=\frac{1}{2}\begin{bmatrix}
	2 & 0 \\0 & 2
\end{bmatrix}
=\begin{bmatrix}
	1 & 0 \\0 & 1
\end{bmatrix}
=I$.
\\
\\
\textbf{Exercise 2.53:} What are the eigenvalues and eigenvectors of $H$?
\\
\textbf{Answer:}\\
$det|A-\lambda I|=det\left|\frac{1}{\sqrt{2}}\begin{bmatrix}
	1 & 1 \\1 & -1
\end{bmatrix}-\begin{bmatrix}
	\lambda & 0 \\0 & \lambda
\end{bmatrix}\right|=
\begin{bmatrix}
	\frac{1}{\sqrt{2}}-\lambda & \frac{1}{\sqrt{2}} \\
	\frac{1}{\sqrt{2}} & -\frac{1}{\sqrt{2}}-\lambda
\end{bmatrix}=0
$\\
Eigenvalues are $λ\pm = \pm 1$ and associated eigenvectors are 
$\ket{\lambda_{\pm}}=\frac{1}{\sqrt{4\pm 2 \sqrt{2}}}\begin{bmatrix}
	1 \\ -1\pm \sqrt{2}
\end{bmatrix}$.
\\
\\
\textbf{Exercise 2.54: } Suppose $A$ and $B$ are commuting Hermitian operators. Prove that \\
$exp(A) exp(B) = exp(A + B).$ (Hint: Use the results of Section 2.1.9.)
\\
\textbf{Answer:}\\
Since $[A, B] = 0,$ $A$ and $B$ are simultaneously diagonalize, $A = \sum_{i} a_{i} \ket{i}\bra{i}, B = \sum_{j} b_{i}\ket{j}\bra{j}$.
\begin{equation}
	\begin{aligned}
		exp(A) exp(B)=&\sum_{i}exp(a_{i}) \ket{i}\bra{i} B = \sum_{j} exp(b_{i})\ket{j}\bra{j} \\
					 =&\sum_{i,j}exp(a_{i}+b_{j})\ket{i}\bra{i}\ket{j}\bra{j}\\
					 =&\sum_{i,j}exp(a_{i}+b_{j})\ket{i}\bra{j}\delta_{i,j} \\
					 =&\sum_{i}exp(a_{i}+b_{i})\ket{i}\bra{i}\\
					 =&exp(A+B).\\
	\end{aligned}
	\end{equation}
	\\
	\\
\textbf{Exercise 2.55: } 
Prove that $U(t_{1},t_{2})$ defined in Equation (2.91) is unitary.
\\
\textbf{Answer:}\\
\begin{equation}
	\begin{aligned}
	U(t_{1},t_{2})^{\dagger}U(t_{1},t_{2})=&exp\left(\frac{iH(t_{2}-t_{1})}{ \hbar}\right)
		\left(\frac{-iH(t_{2}-t_{1})}{\hbar}\right) \\
		=&exp\left(\frac{i\sum_{E_{1}}E_{1}\ket{E_{1}}\bra{E_{1}}(t_{2}-t_{1})}{\hbar}\right)
		exp\left(\frac{-i\sum_{E_{2}}E_{2}\ket{E_{2}}\bra{E_{1}}(t_{2}-t_{1})}{\hbar}\right)\\
		=&\sum_{E_{1},E_{2}}exp\left(\frac{iE_{1}(t_{2}-t_{1})}{\hbar}\right)\ket{E_{2}}\bra{E_{2}}
		exp\left( \frac{-iE_{1}(t_{2}-t_{1})}{\hbar}\right)\ket{E_{1}}\bra{E_{1}} \\
		=&\sum_{E_{1},E_{2}}exp\left(\frac{i(E_{1}-E_{2})(t_{2}-t_{1})}{\hbar}\right)\ket{E_{1}}\bra{E_{1}}\ket{E_{2}}\bra{E_{2}} \\
		=&\sum_{E_{1},E_{2}}exp\left(\frac{i(E_{1}-E_{2})(t_{2}-t_{1})}{\hbar}\right)\ket{E_{1}}\bra{E_{2}}\delta_{E_{1},E_{2}} \\
		=&\sum_{E_{1}}\ket{E_{1}}\bra{E_{1}}\\
		=&I.
	\end{aligned}
	\end{equation}
Thus $U(t_{1},t_{2})$ is unitary.
\\
\\
\textbf{Exercise 2.56: } Use the spectral decomposition to show that $K ≡ −i \log(U )$ is Hermitian for any unitary $U$, and thus $U = exp(iK)$ for some Hermitian $K$.
\\
\textbf{Answer:}\\
Since $ U$ is unitary, then $U$ can perform spectral decomposition, $U=\sum_{i}\lambda_{i}\ket{i}\bra{i}$
\begin{equation}
	\begin{aligned}
K^{\dagger} = &(−i\log(U))^{\dagger} \\
			= &(-i\log(\sum_{i}\lambda_{i}\ket{i}\bra{i}))^{\dagger}\\
			= &(i\sum_{i}\log(\lambda_{i})\ket{i}\bra{i}). \\
\end{aligned}
\end{equation}
\\
\\
\textbf{Exercise 2.57: (Cascaded measurements are single measurements) } 
Suppose ${L_{l}}$ and ${M_{m}}$ are two sets of measurement operators. Show that a
 measurement defined by the measurement operators ${L_{l}}$ followed by a measurement
  defined by the measurement operators ${M_{m}}$ is physically equivalent to a single 
  measurement defined by measurement operators ${N_{lm}}$ with the representation
   $N_{lm} = M_{m}L_{l}.$
\\
	\textbf{Answer:}\\
	If the state of the quantum system is $\ket{\psi}$ immediately before the measurement. 
    The state of the system after the first measurement is 
	$\ket{\psi_{L}}=\frac{L_{l}\ket{\psi}}{\sqrt{\bra{\psi}L_{l}^{\dagger}L_{l}\ket{\psi}}}$ 
	and the second measurement is
	$\ket{\psi_{M}}=\frac{M_{m}\ket{\psi_{L}}}{\sqrt{\bra{\psi_{L}}M_{m}^{\dagger}M_{m}\ket{\psi_{L}}}}$.
		\begin{equation}
			\begin{aligned}
				\bra{\psi_{L}}=&\frac{\bra{\psi} L_{l}^{\dagger}}{\sqrt{\bra{\psi}L_{l}^{\dagger}L_{l}\ket{\psi}}}\\
				\ket{\psi_{M}}=&\frac{M_{m}\ket{\psi_{L}}}{\sqrt{\bra{\psi_{L}}M_{m}^{\dagger}M_{m}\ket{\psi_{L}}}} \\
						  =&\frac{M_{m}\frac{L_{l}\ket{\psi}}{\sqrt{\bra{\psi}L_{l}^{\dagger}L_{l}\ket{\psi}}}}
						  {\frac{\bra{\psi} L_{l}^{\dagger}}{\sqrt{\bra{\psi}L_{l}^{\dagger}L_{l}\ket{\psi}}}
						  M_{m}^{\dagger}M_{m}
						  \frac{L_{l}\ket{\psi}}{\sqrt{\bra{\psi}L_{l}^{\dagger}L_{l}\ket{\psi}}}
						  } \\
						  =&\frac{M_{m}L_{l}\ket{\psi}}{\bra{\psi}L^{\dagger}M_{m}^{\dagger}M_{m}L_{l}\ket{\psi}}.\\
			\end{aligned}
		\end{equation}
		The state of the system after the measurement operators ${N_{lm}}$ ($N_{lm} = M_{m}L_{l}.$) is
		\begin{equation}
			\begin{aligned}
				\ket{\psi_{N}}=&\frac{N_{lm}\ket{\psi}}{\sqrt{\bra{\psi}N_{lm}^{\dagger}N_{lm}\ket{\psi}}} \\
							  =&\frac{M_{m}L_{l}\ket{\psi}}{\sqrt{\bra{\psi}L_{l}^{\dagger}M_{m}^{\dagger}M_{m}L_{l}\ket{\psi}}}
							  =&\ket{\psi_{M}}.
			\end{aligned}
		\end{equation} 
		Thus we proved that Cascaded measurements are single measurements.
	\\
\\
\textbf{Exercise 2.58: } Suppose we prepare a quantum system in an eigenstate
 $ \ket{\psi}$ of some observable $M$ , with corresponding eigenvalue $m$. 
 What is the average observed value of $M$, and the standard deviation?
 \\
\textbf{Answer:}\\
\begin{equation}
	\begin{aligned}
		\left \langle M\right\rangle=&\bra{\psi}M\ket{\psi}\\
									=&\bra{\psi}m\ket{\psi} \\
									=&m\bra{\psi}\ket{\psi} \\
									=m
		[\Delta M]^{2}=&\left \langle M^{2}\right\rangle-\left \langle M\right\rangle^{2}\\
					  =&\bra{\psi}m^{2}\ket{\psi}-m^{2} \\
					  =&m^{2}-m^{2} \\
					  =&0.
	\end{aligned}
\end{equation} 
\\
\\
\textbf{Exercise 2.49: } Suppose we have qubit in the state $\ket{0}$, and we measure the observable $X$. 
What is the average value of $X$? What is the standard deviation of $X$?
\\
\textbf{Answer:}\\
\begin{equation}
	\begin{aligned}
		\left \langle X\right\rangle=&\bra{0}X\ket{0} \\
									=&0\\
		\left \langle X^{2}\right\rangle=&\bra{0}X^{2}\ket{0} \\
									=&1\\
		[\Delta X]=&\sqrt{\left \langle X^{2}\right\rangle-\left \langle X\right\rangle^{2}}=1.				
	\end{aligned}
	\end{equation}
\\
\\
\textbf{Exercise 2.50: } Show that $v\cdot \sigma$ has eigenvalues $\pm1$, 
and that the projectors onto the corresponding eigenspaces are given
 by $P_{\pm} = (I \pm \vec{v}\cdot \vec{\sigma})/2$.
 \\
\textbf{Answer:}	 \\
\begin{equation}
	\begin{aligned}
\vec{v}\cdot\vec{\sigma}=&\sum_{i=1}^{3}v_{i}\sigma_{i}\\
						=&v_{1}\begin{bmatrix}
							0 &1 \\1 & 0
						\end{bmatrix} \\
						=&v_{2}\begin{bmatrix}
							0 &-i \\i & 0
						\end{bmatrix}\\
						=&v_{3}\begin{bmatrix}
							1 &0 \\0 & -1
						\end{bmatrix} \\
						=&\begin{bmatrix}
							v_{3} & v_{1}-iv_{2} \\
							v_{1}+iv_{2} & -v_{3}
						\end{bmatrix} \\
	det(\vec{v}\cdot\vec{\sigma}-\lambda I)=&(v_{3}-\lambda)(-v_{3}-\lambda)
										  -(v_{1}-iv_{2})(v_{1}+iv_{2}) \\
										  =&\lambda^{2}-(v_{1}^{2}+v_{2}^{2}+v_{3}^{2}) \\
										  =&\lambda^{2}-1.
	\end{aligned}
	\end{equation}
	Eigenvalues are $λ = \pm1.$
	if $\lambda=1$
	\begin{equation}
		\begin{aligned}
			\vec{v}\cdot\vec{\sigma}-\lambda I=& \vec{v}\cdot\vec{\sigma}-I\\
											  =&\begin{bmatrix}
												  v_{3}-1 & v_{1}-iv_{2}\\
												  v_{1}+iv_{2} & -v_{3}-1
											  \end{bmatrix} \\
	\end{aligned} 
	\end{equation}
	Normalized eigenvector is $\ket{\lambda_{1}}=\sqrt{\frac{1+v_{3}}{2}}
	\begin{bmatrix}
		1 \\ \frac{1-v_{3}}{v_{1}-iv_{2}}
	\end{bmatrix}$.
	\begin{equation}
		\begin{aligned}
			\ket{\lambda_{1}}\bra{\lambda_{1}}=&\frac{1+V_{3}}{2}\begin{bmatrix}
				1 \\ \frac{1-v_{3}}{v_{1}-iv_{2}}
			\end{bmatrix}
			\begin{bmatrix}
				1 & \frac{1-v_{3}}{v_{1}+iv_{2}}
			\end{bmatrix} \\
			=&\frac{1+v_{3}}{2} \begin{bmatrix}
				1 & \frac{v_{1}-iv_{2}}{1+v_{3}}\\
				\frac{v_{1}+iv_{2}}{1+v_{3}} & \frac{1-v_{3}}{1+v_{3}} 
			\end{bmatrix}\\
			=&\frac{1}{2}\begin{bmatrix}
				1+v_{3} & v_{1}-iv_{2} \\ v_{1}+iv_{2} & 1-v_{3}
			\end{bmatrix}\\
			=& \frac{1}{2}\left(I+\begin{bmatrix}
				v_{3} & v_{1}-iv_{2} \\v_{1}+iv_{2} & -v_{3}
			\end{bmatrix}\right)\\
			=& \frac{1}{2}(I+\vec{v}\cdot\vec{\sigma}).
		\end{aligned}
	\end{equation}
	If $\lambda=-1$. \\
	Normalized eigenvalue is$\ket{\lambda_{-1}}=\sqrt{\frac{1-v_{3}}{2}}\begin{bmatrix}
		1 \\ -\frac{1+v_{3}}{v_{1}-iv_{2}}
	\end{bmatrix}$. \\
	Similarly, we can get the $\ket{\lambda_{-1}}\bra{\lambda_{-1}}=\frac{1}{2}(I-\vec{v}\cdot\vec{\sigma})$.
\end{quote}

\end{document}