\documentclass[UTF8]{ctexart}
\usepackage{physics}
\usepackage{amsmath}
\usepackage{geometry}
\usepackage{indentfirst} 


 \setlength{\parindent}{2em}
\geometry{a4paper,scale=0.8}
\begin{document}
\begin{quote}
\textbf{1 }对于任意一个$m*n$阶的矩阵$A$,若定义
\begin{equation}
	\begin{aligned}
		||A||=&\sum_{i=1}^{m}\sum_{j=1}^{n}|a_{ij}|
	\end{aligned}
\end{equation} 
证明它是矩阵A的范数。
\\
\textbf{解:}\\
证明:
	需要证明此定义满足矩阵范数的四条性质即可。\\
	非负性:$||A||$大于等于0,仅当$A=0$时,$||A||=0$,所以非负性成立。\\
	齐次性:k为任意复数,
\begin{equation}
	\begin{aligned}
		||kA||=&\sum_{i=1}^{m}\sum_{j=1}^{n}|k\cdot a_{ij}|\\
			  =&k\sum_{i=1}^{m}\sum_{j=1}^{n}|a_{ij}|\\
			  =&k||A||
	\end{aligned}
\end{equation} 
三角不等式:\\
\begin{equation}
	\begin{aligned}
		||A||+||B||=&\sum_{i=1}^{m}\sum_{j=1}^{n}|a_{ij}|+\sum_{i=1}^{m}\sum_{j=1}^{n}|b_{ij}|\\
			 \geq &\sum_{i=1}^{m}\sum_{j=1}^{n}|a_{ij}+b_{ij}|\\
			 =& ||A+B||\\
			\end{aligned}
		\end{equation}
相容性:设$A$是$m*p$阶矩阵,$B$是$p*n$阶矩阵,\\
\begin{equation}
	\begin{aligned}
			 ||AB||=&\sum_{i=1}^{m}\sum_{j=1}^{n}\left|\sum_{k=1}^{p}a(i,k)
		\cdot b(k,j)\right|\\
		\leq &\sum_{i=1}^{m}\sum_{j=1}^{n}\sum_{k=1}^{m}|a(i,k)||b(k,j)|\\
		\leq & \sum_{i=1}^{m}\sum_{j=1}^{n}
		\left[\left(\sum_{k=1}^{p}|a(i,k)|\right)
		\left(\sum_{k=1}^{p}|b(k,j)|\right)\right]\\
		=&\left(\sum_{i=1}^{m}\sum_{k=1}^{p}|a(i,k)|\right)
		  \left(\sum_{j=1}^{n}\sum_{k=1}^{p}|b(k,j)|\right)\\
		=&|A||B| \\
	\end{aligned}
\end{equation}
所以此定义是矩阵$A$的范数。
\\
\\
\\
\\
\\
\\
\\
\\
\\
\\
\\
\textbf{2 } 用列主元高斯消去法求解方程组
\begin{equation}
	\left\{
		\begin{aligned}
			3x_{1}+5x_{2}+4x_{3}=&-1\\
			5x_{1}+7x_{2}+3x_{3}=&2 \\
			4x_{1}+4x_{2}+2x_{3}=&2
		\end{aligned}
	\right.
\end{equation}
\\
\textbf{解:}\\
\begin{equation}
	\begin{aligned}
		&
		\begin{bmatrix}
			3 &5& 4\\ 5 & 7 & 3\\ 4 &4 &2
		\end{bmatrix}
		\begin{bmatrix}
			-1 \\ 2 \\2
		\end{bmatrix}
		\underrightarrow{\text{交换第一行和第三行}}\\
		&\begin{bmatrix}
			5&7& 3\\ 3 & 5 & 4\\ 4 &4 &2
		\end{bmatrix}
		\begin{bmatrix}
			2 \\ -1 \\ 2
		\end{bmatrix}
		\underrightarrow{\text{r2-3/5*r1, r3-4/5*r1}}\\
		&\begin{bmatrix}
			5&7& 3\\ 0 & 4/5 & 11/5\\ 0 &-8/5 &-2/5
		\end{bmatrix}
		\begin{bmatrix}
			2 \\ -11/5 \\ 2/5
		\end{bmatrix}
	\underrightarrow{\text{交换第二行和第三行}}
	\\
	&\begin{bmatrix}
		5&7& 3\\ 0 &-8/5 &-2/5\\ 0 & 4/5 & 11/5
	\end{bmatrix}
	\begin{bmatrix}
		2\\ 2/5 \\ -11/5 
	\end{bmatrix}
	\underrightarrow{\text{r3+1/2*r2}}
\\
&\begin{bmatrix}
	5&7& 3\\ 0 &-8/5 &-2/5\\ 0 & 0 & 2
\end{bmatrix}
\begin{bmatrix}
	2 \\ 2/5 \\ -2
\end{bmatrix}
	\end{aligned}
\end{equation}
 回代可求得x1=1,x2=0,x3=-1。
\end{quote}

\end{document}