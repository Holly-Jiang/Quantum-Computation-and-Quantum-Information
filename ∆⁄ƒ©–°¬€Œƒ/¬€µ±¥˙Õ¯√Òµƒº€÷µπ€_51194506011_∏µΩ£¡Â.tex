\documentclass[UTF8]{ctexart}
\begin{document}
	\title{\textbf{论当代网民的价值观}\\[1ex]\begin{large}
			——分析乔碧萝/韩美娟事件
		\end{large}}
	\author{软件工程学院 51194506011 傅剑铃}
	\maketitle
\begin{abstract}
	二十世纪七十年代,互联网率先在西方发达国家出现并逐渐发展;九十年代,互联网慢慢普及西方国家乃至全世界;到二十一世纪初,随着电脑硬件设施成本的降低、盗版系统的爆发兴起、各类电脑应用的开发,互联网深入到了人们的日常生活中,互联网时代真正到来,随之而来的是互联网信息的爆炸。在当下,虚拟网络实践活动在每个人的生活中不断深入渗透,研究其所承载的网民实践行为、文化价值意义以及反映出的网民价值观也日渐凸显出重要性。虚拟网络作为现实生活实践的延伸载体,既继承了客观现实世界的实在性和基础性,又有其自身的独特之处;虚拟网络的文化价值内容,既包括了网络文化实践过程中创造出的文化特质,又包括网络文化主体——网民本身所具备的社会文化价值观,是对社会价值、社会经验评判价值、社会和谐互动和社会理性价值等诸多方面充分而又集中的体现。只有深入了解当代网民自身的社会文化价值观和研究其虚拟实践活动背后的意义,才能更好地把握网络价值文化的本质,在网络文化融合和产生冲突问题的时候,及时对一些扭曲、偏离的社会文化价值观进行纠正和引导,才能更好地践行虚拟网络作为客观现实世界延伸载体的作用,才能更好地使社会整体价值观朝主流方向发展、凝聚打造更和谐的社会。\\
	\\
	\textbf{关键词:}虚拟网络文化价值 \, 社会文化价值观 \,  网络文化主体 \, 网民价值观
\end{abstract}
	
\section{背景介绍}
	随着微博、抖音、快手等各大网络社交平台的出现,各式各样的网络信息爆炸式地渗透到了每个网民的日常生活之中。据国家统计局发布的经济社会发展报告中显示,我国网民由1997年的62万人激增至2018年的8.3亿人,网络使得人际交往、言论发表、信息传递的门槛逐渐降低,只需要一台电子通信设备和入网宽带,人人都可参与其中,主动或被动地接受网络文化以及文化背后所传递的社会价值观。在网络时效性高,传播速度快、范围广的特点下,人们可以在大量网络热点中畅所欲言,来自不同背景、不同教育程度、不同文化习俗的人们在同一平台上进行网络实践活动,在创造网络文化价值内容的同时,也在传递着网民本身所携带的社会价值观,网络既承载着网民的价值理想,却也同时影响着网民的价值判断。因此,网络文化自身所具有积极作用和负面影响的两面性,和网络主体——网民价值观的多样性相融合,使得网络环境纷繁复杂,大量负面非主流的评价的出现,正在潜移默化地影响着网络本应传递的文化价值观,使部分人出现了正确价值观缺失现象\cite{LYX}。
	\par{2019年7月,斗鱼平台发生了轰动一时的“乔碧萝殿下直播翻车”事件,“萝莉变大妈”一时成为占据各大平台榜首的热点话题,被惹恼的网友们纷纷表示这是欺诈行为,甚至人肉威胁、辱骂攻击涉事主播。事后,平台发布公告表示,事情系主播恶意策划炒作,挑战公众底线,对社会造成了不良影响。同一时期,蝴蝶姑娘韩美娟在抖音平台突然爆红,“百因必有果,你的报应就是我”一时成为人们口耳相传的“金句”,希望依靠互联网挣钱养活自己跟相依为命的奶奶的韩佩泉,本可以传播励志的形象,却因为其夸张的妆容形象,被网友们用恶心、恐怖、卖惨等字眼肆意抨击。在网红不断层出的时代,网民们对网红的选择从一定程度上代表了当代中国网民社会价值观的缩影。}
\section{当代网民社会价值观分析}
	\subsection{什么是网络文化价值?}
	什么是网络文化?网络文化是以网络为平台,以计算机和通信技术为支撑,以数字化为表现形态,对人们的生产生活方式及其价值观念产生影响的文化形态\cite{WN}。而网络文化价值,也就是在这种新型生产生活方式下,网络文化和网络主体——网民之间需要和被需要的关系,以及这种新兴文化形态对人类及其社会所带来的的影响和作用。网络作为现实世界的延伸平台,它既传承了现实世界所有实践活动所具有的性质,同时,它又保持着自身虚拟的特性,网络文化世界是一个以虚拟现实为背景和空间的文化世界,由人们在此虚拟实践过程中创造的具有文化特质的载体。因此,实践效应的正负面性质决定了网络文化价值的两面性,即:网络文化的正面价值和其负面价值。
	\subsection{什么是网民社会价值观?}
	网民作为网络文化主体,既是网络实践活动的发起者,同时也是网络实践活动的承受者;网民的价值观构造了网络世界的内在逻辑并引导了网络社会的发展方向。性别、年龄、受教育程度等人类学社会学特性存在差异,使得网民的价值观丰富多元化。因此,网民对网络文化的价值选择,反映出了主流倾向、大众倾向和先锋倾向\cite{TKY}。作为网络实践活动的发起者,网民的网络行为是由一定的价值观制导的;而作为承受者,网络实践活动的过程中,网民的价值观也在潜移默化受到不同价值观的冲击和影响。
	\par{从另一角度来看,网民价值观是网民对于网络行为、网络问题的好与坏的价值认识和价值意识识别;多元性、多层次的网络文化结构深层上反映出了网民不同的价值意识、思维方式和世界观。网络文化和网民价值观相互依赖、相互作用,网络为人类创造了新的文化,为人类创造了新的文化载体\cite{WN},在长期网络文化交流的过程中,文化又由网民的价值观作出选择而达到相对稳定的程度。}
\section{当代网民价值观偏离原因探究}
	\subsection{价值观念的转变}
	价值观,主要是围绕着群己、义利、理欲、德力等关系问题而展开的,对前后两者不同的侧重形成了对价值观问题的争论\cite{CZL}。改革开放以来,在市场经济的冲击下,中国百姓的生活方式发生了巨大的改变,随之加剧了人们之间不同价值观念的冲突。在西方文化大量涌入的背景下,网络平台同时增加了不同文化传播的速度和范围,使人们传统的价值观念产生了动摇,市场经济制度下的利益激励机制刺激了人们心中对利益、欲望的追逐,享受主义和利己主义带来的及时快感冲击着“重义轻利”的传统价值理念,甚至在一些意志薄弱的人身上出现了盲目崇拜和跟风的“崇洋媚外”现象。
	\subsection{多元媒体的冲击}
	在以往纸质传媒的日子里,一篇文章从撰写、编辑审核、印刷到面世传播,需要花费较大的时间和精力,并且需要通过层层严格的把控;而在数字多媒体的时代,有了网络这一媒介,消息的传播门槛变得非常低,任何人只需一个马甲(网络身份)和上网工具,便可以随时随地编辑发送。因此,在门槛降低的条件下,网络消息的真实性、可靠性和深度性变得有待商榷,同时多元媒体技术使得消息能够图文并茂,一些欠缺社会阅历、文化程度较低、理论根基浅薄的网民往往容易陷入信息的误导,从而产生对自我道德约束和责任观念的怀疑中,造成正确价值观的偏离和缺失。
	\subsection{虚拟网络的影响}
	网络是现实世界的延伸,然而网络又具有虚拟的这一特性。这就导致:活跃在网络上的人们对于彼此的背景信息一无所知,同时也因自己身份隐匿而存有一丝侥幸心理,将自我与现实世界切割开来,挣脱了现实世界的限制和束缚。网络作为无需承担责任的舆论阵地,人们可以自由发表言论、发泄情绪,长此以往,人们在这种“自由”的状态下容易混淆是非的价值观念,颠覆传统的价值道德标准\cite{LJY}。
	\subsection{理想信念的缺失}
	纵观诸多外界因素,归根结底,根本原因在于人们内心的理想信念不够坚定,在外界文化的冲击下,容易产生对原有理想信念的怀疑,这种怀疑产生迷茫和困惑,最终导致人们理想信念“摇摆不定”和价值观上的变化。
\section{应对策略}
	\subsection{确定网络文化建设目标}
	要以培育有理想、有道德、有文化、有纪律的社会主义网民为网络文化建设的主要目标\cite{MST}。网络文化需要发挥其积极正面的作用,团结统一人们的思想,凝聚人心;网络文化要努力提高全网民的思想道德和科学文化素质,营造健康良好的网络人文环境;既要满足人民群众日益增长的精神文化需求,也要严格把控精神食粮,去粗取精,把符合社会主流、正确的文化素材奉献给广大人民网友。
	\subsection{创新思想政治教育方法}
	借助网络平台交互性、开放性、平等性、创造性、高效性等特性\cite{YWG},着力于网络思想政治教育平台的优化,利用好开展开放性的网络平台,及时抓住网民的思想动态,展开针对性教育;根据网民遇到的不同问题具体展开分析,从根本入手研究网民的诉求,从根本上解决网民的问题;利用网络渗透性,在潜移默化中达到事半功倍的思想政治教育\cite{LYX}。
	\par{有效利用网络信息的更新速度快和网民热衷的特点优化思想政治教育工作环境,打破传统的教育内容和方法的限制,传播主流价值观;整合网络中参差不齐的内容,净化网络平台的不良思想,营造良好的网络环境。}
	\subsection{健全网络监督管理机制}
	不少人持有“网络不需要负责任”的错误思想,因此肆意妄为地无视公共规则,宣扬自己病态的价值观,蔑视侮辱他人,发表不正确的言论。但世界上不存在任何法外之地,完善网络平台监管机制,加强对网络平台的监督管理,每个网络平台都应该制定相应的网络文明公约,提醒网民遵守规则,文明上网,并对网上违法行为予以严厉处罚。
	\par{制裁无良行为的实质上就是对其背后的错误价值观给予否定,这有利于抑制错误价值观的蔓延,促进健康价值观的传播。只有网络监管部门、网络平台和网民三者齐心协力才能营造一个良好的网络环境,才能为网络时代人们价值观的形成提供积极正面的影响。\cite{HZF}}
	
% ---- Bibliography ----
\bibliographystyle{splncs03}
\bibliography{cite}

\end{document}