\documentclass[UTF8]{ctexart}
\begin{document}
	\title{\textbf{应试教育与素质教育结合我国国情分析}\\[1ex]\begin{large}
		\end{large}}
	\author{软件工程学院 51194501050 蒋慧}
	\maketitle
\begin{abstract}
	经过二十多年的改革开放和发展,我国的社会生产力、综合国力和人民生活水平都上了一个大平台,市场供求关系、体制环境和对外经济关系发生重大变化,为开始实施现代化建设第三步战略部署奠定了良好的基础。进入二十一世纪,经济全球化趋势增强,科技革命迅速发展,产业结构调整步伐加快,国际竞争更加激烈。世界经济形势的深刻变化和发展趋势,特别是加入世界贸易组织,将给我国带来新的发展机遇和严峻挑战。驾驭是提高全民素培养人才的基础,要面向现代化、面向世界、面向未来,适度超前发展,走改革创新之路。人才是最宝贵的资源,要把培养、吸引吸引和用好人才作为一项重大的战略任务切实抓好,按照德才兼备的原则,培养数以亿计高素质的劳动者、数以千万计具有创新精神和创新能力的专门人才。教育是为国家服务的,教育就是为国家培养有用人才,国家需要什么样的人才,就培养什么样的人才。缺技能人才,就培养技能人才,缺创新型人才,就培养创新型人才。国家实行素质教育目的就是指依据人和社会发展的实际需要,通过优质的考核机制,培养国家需要的优质人才,提高国家竞争力。应试教育应是素质教育重要组成部分,当然不是说不需要对现行“应试考试”体制进行改革了。国家提倡素质教育说明素质教育是符合党的教育方针的。这就要求现行教育体制必须作出改革,以适应国家发展的需要,适应社会发展的需要。现行的“应试考试”体制是必须进行改革的,关键是改革要掌握一个“度”\textsuperscript{\cite{CZJ}}。如不能过分布置太多的作业,教材内容按照素质教育以能力为核心的原则去规划设计、教学要以人为本因材施教、设计合理的评价制度、解决教育资源的不平衡性等。通过改革使应试教育成为一种良性的应试教育,让这种良性的应试教育真正变成素质教育的一部分,实现教育的真正的目的。\\
	\\
	\textbf{关键词:}国民经济 \, 国民素质 \,  应试教育\, 素质教育
\end{abstract}
\section{背景介绍}
	国家各级各类教育着力推进素质教育,重视培养创新精神和实践能力,促进学生德智体美全面发展。把加强基础教育放在重要位置,继续提高国民教育普及程度。然而由于当前经济发展不平衡,教育资源分配不均衡,为了获得更高的教育水平,高考作为应试教育应运而生,实施高考制度也是由我国的基本国情所决定的,因为我国是一个人口众多的国家,经济发展不稳定导致我国不能给广大的学生提供太多的学习机会和学习场所,因此只能通过高考制度考试录取优秀者继续学习,对于我国的国情而言,高考制度是相对比较公平、公正、公开的考试方式。应试教育是指以追求考试分数和升学率, 忽视人的全面发展和社会发展需要的教育。 应试教育体制带来诸多弊端,国内学者关于此类的研究也较多,尤其是应试教育对中小学生学习的影响研究比较多,但是应试教育对于大学生学习影响的研究比较少。 大学教育不需要追求升学率,学校对于学生的考试分数也没有中学那么重视,虽然也有一些研究指出一些大学片面追求考研率,大学里普遍存在大学生热衷于出国留学考,公务员考试,各种证书考试现象以及大学教育教学方式和管理体制上的问题也使得大学教育存在应试教育倾向,但总体来讲,大学里的应试教育氛围没有中学那么浓厚。 然而,长年在应试教育体制下培养出来的高中生,尽管进入到学习氛围相对轻松自由的大学校园,他们在学习的各个方面还是受到应试教育的影响而表现出一些典型的学习心理特点。 大学生学习心理渗透到学习的各个方面,不仅影响着大学生是否能够在大学里更好地获取知识,促进自己的全面发展,还影响到大学生是否很好地成长成才,为社会发展作出贡献\textsuperscript{\cite{CXF}}。
	\par{在高考制度下,学校教育目的非常明确,就是提高学生分数,提高一本二本上线率。 在家里,家长对孩子的要求是“只要好好读书考上大学就可以了, 其他什么都不用管”。 因此,学生学习目的也非常明确,就是考上大学。这种长期来自学校和家长对学生的一种功利化的学习目的灌输也影响到大学生对于学习目的和学习意义 的正确认知。应试教育体制下,学校教育内容单一,学生围绕考试而学,教师围绕考试而教,忽视了学 生的情感需要、认知需要、审美需要。学校为高考制定了一系列套路,教师研究考试大纲,研究习题,帮学生划重点,做笔记,把 有利于提高学生成绩的所谓的重点知识灌输给学生, 学生只 管识记,在题海战术中提高自己的解题能力,提高卷面成绩。大学与高中最大的不同就 是每天上课的时间少了, 大学生有很多自己自由支配的时 间。 但是一些大学生由于习惯了高中每天被安排满满的学习任务,到了大学就不会自主安排课余学习时间,加上没有老师和家长的督促,逐渐放松了学习,有的学生甚至出现逃课、 挂科的现象。
}
\section{应试教育出现原因利弊分析}
	\subsection{什么是应试教育?}
	应试教育是一种以应付升学考试 为唯一目的 , 围绕“应考”开展教育教 学活动的片面的、淘汰式教育”。当前 中小学教育仍然以应试教育为主,仍然追求升学率,忽视了学生身心发展的整体性,过于注重对知识的掌握,忽视了 学生其他方面能力的发展,不利于学生的全面发展。应试教育也忽视人身心发展的差异性,对学生的发展一刀切, 用同一标准来要求学生,不利于学生个性的发展\textsuperscript{\cite{XQ}}。
	\subsection{应试教育与素质教育对比分析}
	\par{应试教育不利于学生的全面发展 体现在以下几个方面:第一,应试教育 过于注重对知识的传授,而忽视了知识 中的道德内容对学生的影响。例如:在 中小学的思想品德课中,教育者只是让 学生记住要考试的相关内容,而没有强 调学习其中的道德准则,更不会要求将 其道德准则内化然后转化为学生的合 乎规范的道德行为,仅仅只把人培养成 擅长考试的机器。第二:应试教育由于 过于强调人对大纲知识的掌握,将人的 学习内容局限于书本之内,导致人的知 识面狭窄,出现高分低能的局面,出现 只会考试而智力低下的人。第三是应试教育对书本知识的过分注重,会导致教育者将大把时间放在受教育者学习知识上,而忽视发展学生的体力和体育运动的技能。}
	\par{素质教育,是指依据人的发展和社会发展的实际需要,是以提高人的基本素 质为根本目的,以尊重人的主体性和主动精神,以人的性格为基础,注重开发人 的智慧潜能,注重能力培养,注重形成人的健全个性,注重人的自我终身可持续 发展为根本特征的一种教育思想。}
	\par{实施素质教育也需要建立相应的考核机制。考试本身只是一种选拔手段,无论社会发展到任何程度,考试是必须的。应试教育存在弊端,加重学生的学习负担;忽视了学生身心发展的整体性和差异性,不利于发展学生的全面发展和个性发展。因此教师,家长,社会要 树立正确的教育理念,实施全面内容的 知识教学,在注重知识教学的过程中, 培养学生的道德、增强体质、培养学生 的审美能力和劳动能力;教师还要根据 学生的个别差异性制定多元化的评价 标准,从而促进学生的发展。国家的发展需要素质教育,同样国家的发展也需要竞争。竞争是社会进 步的推动力,只有竞争才能激发主动性和积极性,促使人的潜力得到充分发挥, 使人们不断进取,奋发向上。应试教育的优点也是素质教育的要求。做好素质教育,先做好应试教 育,二者并不冲突。应试能力也是孩子能力的一部分,应试素质也是素质。应试 教育本身可以锻炼孩子的领悟能力、记忆能力,理解能力,判断能力,表达能力 和专心细致能力等。这一切何尝不是孩子的素质、能力?应试教育应是素质教育重要组成部分,当然不是说不需要对现行“应试考 试”体制进行改革了。国家提倡素质教育说明素质教育是符合党的教育方针的。 这就要求现行教育体制必须作出改革,以适应国家发展的需要,适应社会发展的 需要。现行的“应试考试”体制是必须进行改革的,关键是改革要掌握一个 “度”\textsuperscript{\cite{CZJ}}。如不能过分布置太多的作业,教材内容按照素质教育以能力为核心的原 则去规划设计、教学要以人为本因材施教、设计合理的评价制度、解决教育资源 的不平衡性等。通过改革使应试教育成为一种良性的应试教育,让这种良性的应试教育真正变成素质教育的一部分,实现教育的真正的目的。}
	\par{应试教育是选拔性的,是精英教育,过分强调教育的单 一功能智力的培养.而素质教育则是面向全体学生,德智体 美劳全面发展的教育;是培养创新精神,促进学生个性发展 的教育;是尊重学生的主体地位、独立人格和造就平等公民 的教育;是教育内容开放传递渠道多样化的教育。\textsuperscript{\cite{LHL}}}
\section{当代国情下国民素质分析}
	\subsection{提高国民素质建设人力资源强国}
	 教育是人力资源开发的主要途径.人力资源建设要主动对接创新型国家战略需要,以推进素质教育为主题,以提高培养质量为核心,形成创新人才培养的良好生态.要以学生为中心,借助科技促进学习方式变革,着力提高学生的实践能力和综合素质;要将特殊人才作为创新型国家人才队伍的重要组成部分,重视特殊人才培养;全面推进现代学徒制培养模式,增强企业在教育和培训中的参与权和获得感;高度重视政府在创新人才培养中的主导作用,以终身学习理念引领创新人才培养体系;加强人才培养的系统规划和顶层设计,建立可持续的人才培养机制,为全面提高国民素质、建设人力资源强国目标奠定坚实基础。中国教育日益被"进身"的功利主义目标化约掉了后者的价值而异变为应试教育.但高校法人主体地位"悬空"下的高考制度,只是应试教育的一个物质载体和推手,绝不是"真凶".应试教育的根子在于以体制内公有部门(国有企业和事业单位)为目标的社会流动机制对以社会市场部门为目标的社会流动机制的冲击,在于前者对教育方向的宰制.只有促进体制外就业部门的大发展才能为社会民众开辟除考公务员事业编、千方百计进国企之外的更多社会上升之渠道,才能消除整个民族身上那种过分高亢的应试兴奋\textsuperscript{\cite{LGD}}。
	\subsection{教育面临的现实}
	当下中国,对于贫寒子弟来说,出现了两种主要的社会流机制:一是通过各种考试,考进体制内的国有企事业单位。 二是去社会市场部门(主要由私营企业和外企组成),在市场 的摸爬滚打中锻造生存的能力实现向上的流动。前者进人的手 段主要就是考试。后者进入的办法主要是看市场检验的效果, 尽管市场体制下老板也会看一个人毕业的院校和学历。而且前者在人均资源的占有量、职业声望(公务员、国企员工)、平 均收入、职业环境、职业稳定程度、职业保障等方面有着显著 的优势,理性地,绝大部分家庭坚决支持学子们接受以考试为主要目标的应试教育进而进好大学、进好国有企事业单位,就 是容易理解了。在当下中国学子就业日渐吃紧的形势下,考上好大学才具有关键性意义。日趋激烈的名校角逐战调动了 全民族的巨大的应试教育的能力、资源、耐心和心情。读书之 进身竞争压得各级学子们身心俱疲,苦累不堪。压力反弹下, 要求素质教育的呼声和愿望也日渐强烈,但对教育公正眼巴巴 渴望的压力使当局不敢轻动千军万马过独木桥的应试教育的核 心——高考。其实准确地说在当下大学教育垄断体制未破除和 工作市场竞争体制未改善之前,对高考的根本性改革,要么是画虎类犬,要么是情况更糟。
	\subsection{国民素质与经济发展的关系}
	国家的发展,社会的进步,生产力的提高决定了国民素质的发展状况\textsuperscript{\cite{MYB}},有什么样的社会生产力就有什么样的社会生产关系,就有什么样的国民素质。 反之,国民素质与人民文明程度的提高在一定程度上也反作用于经济社会的发展。我们应该正确认识国民素质对于经济发展的双向作 用,大力发展教育事业,深度挖掘传统文化精髓,树立正确的发展 理念,为新时代经济社会的发展注入活力和动力。
\section{应对策略}
	\subsection{确定高校培养目标目标}
	树立多元化人才观,鼓励大学生创业:高等教育应树立多元化的人才观,切实领会我国教育目的精神,大力发展素质教育,要根据学生的个性特点,培养具有创新精神、实践能 力和独立个性的社会主义现代化需要的各级各类人才能适 应社会对人才的多样化需求。加强对大学生学习心理的辅导和教育:良好的学习心理,需要对学习有正确的认知、适当的学习动机、好的学习方法, 还需要对自己的学习有合理的计划和安排, 有一定的自制力\textsuperscript{\cite{CXF}}。
	\subsection{建设教育强国}
	建设教育强国是中华民族伟大复兴的基础工程,必须把教育事 业放在优先位置,深化教育改革,加快教育现代化,办好人民满意的教育。一个国家、一个民族的强盛,总是以文化的兴盛为支撑。我国拥有五千年的历史发展进程,传统文化悠久而深厚。新的时代背景下,我们应根据历史发展的新潮流,树立正确的 发展观念。
	\subsection{高校制度改革}
	破除公立高校一统天下,一家独大的教育格局,大力培育各类办学主体,形成以民办为主,公办为辅的竞 争性教育格局。只有这样,才能让大学真正基于自身生死存亡 的压力下,为学生负责,高效、优质地办学,公正地选拔新生, 制定符合人的发展规律的选拔导向标准而不是一考定乾坤。才 能在制度上消减基础教育和中学教育的应试冲动,而不是只为大学输送应试尖子。才能真正还原初级和中等教育自身读书修 身长智的本初性质。
	
% ---- Bibliography ----
\bibliographystyle{unsrt}
\bibliography{cite}

\end{document}